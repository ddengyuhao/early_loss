%!TEX root = ../main.tex
\section{Preliminary} 
\label{sec:pre}

\subsection{Problem Definition}

\stitle{Detection with parameter(gradient?) approximation}

The setting of our problems.

Input: A training set which contains mislabeled data.
% \begin{itemize}
%	\item A trainging set which contains mislabeled data.
% \end{itemize}

Output: A subset of the initial training set whose parameters are closest to those of the model trained by removing all the real dirty data in the training set.

\stitle{Complexity analysis:} NP

\subsection{Early Loss}

The accuracy on ``clean'' samples is higher than on the ``bad'' samples, especially in the initial epochs of training. In other words, the training loss on correctly labeled instances is higher than that of the incorrectly labeled instance.

\subsection{Parameters Influence}

Parameters of the model which is trained by a dataset only contains ``clean'' samples are more likely to have a larger change when adding one ``bad'' sample into the dataset, compared with adding a ``clean'' sample.