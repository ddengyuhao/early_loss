%!TEX root = ../main.tex
\section{Preliminary} 
\label{sec:pre}

In this Section, we will first introduce some basic concepts, followed by  the problem definition.

\stitle{Mislabel detection.} Given a dataset $\train=\{o_1, o_2, ..., o_N\}$, each instance  $o_i = (x_i, y_i), i\in [1,N]$ is associated with the feature $x_i$ and a corresponding label $y_i$. Labels are error-prone due to many reasons like crowdsourcing workers' error, so each instance has an unknown ground truth label $y_i^*$. The problem of mislabel detection is to identify all mislabeled instances, \ie $\train_m = \{o_i|o_i\in \train, y_i \neq y_i^*\}$. Besides, we use $\train_c$ to denote the set of clean instances, \ie $\train = \train_c \cup \train_m$.
Note that  we could  support any data type, and in practice, we test images and tabular data.

\stitle{Supervised machine learning.} As we all know, labels play a significant role in supervised ML training.
Clean training instances will no doubt lead to a smooth training process and better ML model, and thus it is reasonable to leverage ML model to detect mislabels~\cite{}. Given a clean dataset, \eg  $\train_c$ , the objective of training  is to compute the best parameter so as to minimize the training loss. To be specific, 

\begin{equation}~\label{eqa:loss}
	w^* = \arg\min\limits_{w\in \mathcal{W}} l(w), l(w)= \frac{1}{|\train_c|}\sum_{i=1}^{|\train_c|} l_i(w, x_i)
\end{equation}

\noindent where $\mathcal{W}$ represents the parameter space, and $l_i(w, x_i)$ denotes the loss of the $i-$th training instance. Typically, the gradient decent algorithm is often applied to optimize Equation~\ref{eqa:loss}. To achieve high efficiency, we can use stochastic gradient decent to accelerate the training process. Note that we can support multi-classification tasks, \ie the label $y$ has to be categorical values. For ease of representation, we can abbreviate $l_i(w, x_i)$ as $l_i(w)$.



\stitle{Our problem definition.} As discussed above, if we decide to use machine learning model to help detect mislabeled instances, we can have the following formulation.

\begin{equation}~\label{eqa:optimize}
\train_m = \arg\min\limits_{\train'\subseteq \train} \rVert w_{\train'}-w^* \lVert
\end{equation}

\noindent where $w_{\train'}$ represents the model parameter trained over the dataset $\train \setminus \train'$. Overall, Equation~\ref{eqa:optimize}
is easy to understand that we aim to detect a set of mislabeled instances $\train'$ such that training over $\train \setminus \train'$ is equivalent to training over $\train_c$. Obviously, when we detect exactly all the mislabeled instances, the above optimization goal will be achieved. However, Equation~\ref{eqa:optimize} is rather challenging to be solved because (1) we cannot obtain $w^*$ because we do not know $\train_c$ in advance, and (2) assuming that we can know  $w^*$, it is prohibitively expensive to enumerate the subsets of $\train$ to find $\train_m$.

Next, we introduce the over framework of \sys to address the mislabel detection problem using ML model.



%\subsection{Early Loss} The accuracy on ``clean'' samples is higher than on the ``bad'' samples, especially in the initial epochs of training. In other words, the training loss on correctly labeled instances is higher than that of the incorrectly labeled instance.

%\subsection{Parameters Influence} Parameters of the model which is trained by a dataset only contains ``clean'' samples are more likely to have a larger change when adding one ``bad'' sample into the dataset, compared with adding a ``clean'' sample.